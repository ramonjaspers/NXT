\label{_Contents}%
 \hypertarget{index_contents_sec}{}\section{Table of Contents}\label{index_contents_sec}
\href{#Introduction}{\tt Introduction}~\newline
 \href{#VersionLog}{\tt Version Log}~\newline
 \href{#Libraries}{\tt Libraries}~\newline
 \href{#CodingStandard}{\tt Coding Standard}~\newline






\label{_Introduction}%
 \hypertarget{index_intro_sec}{}\section{Introduction}\label{index_intro_sec}
This document serves the purpose of providing the background and details of the embedded software that was developed for the TI group project.

The main purpose of this project was to automate a robot which drives on 4 wheels to follow a line with specific tasks on his way to the end.~\newline
 The robot must be able to follow a line by himself.

~\newline
\href{#Contents}{\tt Table of Contents}~\newline






\label{_VersionLog}%
\hypertarget{index_Version}{}\section{Version Log}\label{index_Version}
\hypertarget{index_Release1}{}\subsection{Release 1}\label{index_Release1}

\begin{DoxyItemize}
\item added Initial Program Structure
\item added Spi Driver
\item added Dio Driver
\item added Task Scheduler
\item added Adc driver
\item added Canbus Driver
\item added C\+AN Application Layer
\end{DoxyItemize}

Final build capable of detecting an event through Adc and storing the event to R\+AM. R\+AM data can be retrieved through the C\+AN bus.

~\newline
\href{#Contents}{\tt Table of Contents}~\newline
 



\label{_Libraries}%
 \hypertarget{index_intro_sec}{}\section{Introduction}\label{index_intro_sec}
List of the used libraries\+: 
\begin{DoxyItemize}
\item a 
\item b 
\item c 
\item d 
\item e 
\item f 
\item g 
\end{DoxyItemize}

~\newline
\href{#Contents}{\tt Table of Contents}~\newline






\label{_CodingStandard}%
\hypertarget{index_CNC}{}\subsection{Code Naming Convention}\label{index_CNC}
Function names should conform to the following standard, ~\newline

\begin{DoxyItemize}
\item Function names should start with a three letter sub-\/system name followed by an under-\/score. ~\newline

\item The first character of the sub-\/system name shall be capitalized. ~\newline

\item The rest of the function name should not have any under-\/scores. ~\newline

\item The rest of the function name should describe what the function does.~\newline

\item The first character of each word should be capitalized. ~\newline

\item Global variables and variables that are static to a module should have their names conform to the same standard as function names. ~\newline

\item Sub-\/system names aren\textquotesingle{}t required, but use them if you can.~\newline
~\newline

\end{DoxyItemize}

For example,
\begin{DoxyItemize}
\item Spi\+\_\+\+Init(\&\+Spi\+\_\+\+Config), where \char`\"{}\+Spi\+\_\+\char`\"{} is the sub-\/system and \char`\"{}\+Init\char`\"{} describes the action to be performed.~\newline
~\newline

\end{DoxyItemize}

Common first words after the sub-\/system name,
\begin{DoxyItemize}
\item \mbox{\hyperlink{class_init}{Init}}
\item \mbox{\hyperlink{class_motion}{Motion}}
\item \mbox{\hyperlink{class_sensor}{Sensor}}
\item \mbox{\hyperlink{class_servo}{Servo}}
\end{DoxyItemize}

For function prototype parameters,
\begin{DoxyItemize}
\item The sub-\/system shall not be used.
\item The first character of each word shall be non-\/capitalized.
\item There should be no under-\/scores in the name.
\item If the name contains more than one word, each word after the first shall have its first character capitalized.
\end{DoxyItemize}

For local variables (to a function),
\begin{DoxyItemize}
\item The sub-\/system shall not be used.
\item The first character of the first word shall not be capitalized.
\item If the name contains more than one word, each word after the first shall have its first character capitalized.
\end{DoxyItemize}

For example,~\newline
~\newline



\begin{DoxyCode}
uint16 Spi\_Read(uint8 *charptr)\{
  uint8 i;
  uint8 myVariable;

  \textcolor{comment}{// Some code}
  statements;
\}
\end{DoxyCode}
~\newline



\begin{DoxyItemize}
\item The parameter \char`\"{}\+Charptr\char`\"{} has the first character un-\/capitalized and does not start with a sub-\/system name because it is a parameter.
\item The variable \char`\"{}i\char`\"{} is not capitalized and does not start with a sub-\/system name because it is a local variable.
\item The variable \char`\"{}my\+Variable\char`\"{} has it\textquotesingle{}s first word un-\/capitalized, it\textquotesingle{}s second word capitalized, and does not start with a sub-\/system name because it is a local variable.
\end{DoxyItemize}\hypertarget{index_CF}{}\subsection{Code Formatting/\+Style}\label{index_CF}
Code formatting and style shall conform to the following guidelines.


\begin{DoxyItemize}
\item Indentation shall be 2 spaces (not tabs).
\item Opening braces shall be placed on the same code lines.
\item Closing braces shall be placed on separate lines.
\item If, for, switch, and while statements shall always use braces.
\item A comment block shall be placed before every function definition describing what the function does, any paramaters passed to it, any return value, and anything else that would be relevant or useful to someone that has to maintain it.
\end{DoxyItemize}

For example, ~\newline
 
\begin{DoxyCode}
\textcolor{comment}{// This is correct.}
\textcolor{keywordflow}{if} (condition)\{
  statement;
\}\textcolor{keywordflow}{else}\{
  statement2;
\}

\textcolor{comment}{// This is <b>NOT</b> correct.}
\textcolor{keywordflow}{if} (condition)
\{
  statement;
\}
\textcolor{keywordflow}{else}
\{
  statement2;
\}

\textcolor{comment}{// Neither this is correct.}
\textcolor{keywordflow}{if} (condition)
  statement;
\textcolor{keywordflow}{else}
  statement2;
\end{DoxyCode}


~\newline
\href{#Contents}{\tt Table of Contents}~\newline
 

 